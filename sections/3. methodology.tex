\documentclass[../main.tex]{subfiles}
\graphicspath{{\subfix{images/}}}

\begin{document}

\section{Methodology}
\subsection{Watching conspiracy videos}

\subsection{Leaving the filter bubble}

\subsection{Machine learning}
\subsubsection{Data gathering}
To answer the research question, it is necessary to determine which YouTube videos can be considered conspiracy videos. Considering the large amount of videos getting recommended, determining each video manually is simply not possible. There are two possible ways to solve this problem. Firstly, there is a dataset which contains nearly 7000 YouTube channels that have been manually labeled based on their politcal view - almost 2500 of which were labeled as conspiracy channels \citep{ledwich2019algorithmic}; whenever a video is made by one such channel, it can be considered a conspiracy video. However, due to the enormous amount of existing YouTube channels, the odds of a video being uploaded by a channel that is not present in this dataset are very large. For those videos, a supervised machine learning classifier was used. To optimize performance, five different classifiers have been trained and compared: k-nearest neighbors, support-vector machine, neural network, logistic regression, and ridge regression. 

In order to train these machine learning algorithms, a training dataset was created. To get a labeled dataset of conspiracy and non-conspiracy videos, use was made of the aforementioned channel dataset made by \citet{ledwich2019algorithmic}. For each channel in that dataset, the title, description, and transcript of the ten most recently uploaded videos were downloaded using YouTube's API. Videos uploaded by a conspiracy channel were then labeled as conspiracy videos, and videos uploaded by a channel from a different category were labeled as normal videos. Additionally, the channel description and channel keywords (which are used for targeted advertising on YouTube) were added to each video. The final dataset contains 65.683 unique YouTube videos, 22.156 of which are considered as conspiracy videos. 

\subsubsection{Data cleaning}
However, this dataset was not yet suitable for machine learning, as the data was still messy. Therefore, multiple steps were taken in order to clean the data. Firstly, the two classes (conspiracy and non-conspiracy) were balanced, so that the classifier would not develop a bias for non-conspiracy videos. Rather than opting for balancing the two classes through the use of class-weights, the choice was made to under-sample the data in order to equalize both classes (both containing 22.156 videos, for a total of 44.312 videos) \citep{lemaitre2017imbalanced}. As there was plenty of data in the dataset, under-sampling was more convenient than implementing class-weights: a technique where weights are attributed to classes, thereby telling the classifier that getting a prediction correct for a certain, underrepresented class is more important \citep{sun}. After both classes had been balanced, the text for each video had to be translated into English. As the original dataset by \citet{ledwich2019algorithmic} also contained channels by non-English speakers, these videos had to be automatically translated. Then, a few common cleaning methods were applied: all text was converted to lowercase, after which special characters, such as emojis were removed, whereafter stop words were removed and all words were stemmed using the porter stemmer \citep{karaa2013new}. Finally, each video was TF-IDF vectorized to allow the classifiers to function.

\subsubsection{Performance optimization}
After splitting the dataset into a train, test, and validation set, the hyperparameters of each algorithm were tuned to get the optimal performance. Performance was measures using four distinct metrics: the accuracy, which shows the share of correct predictions; the recall, which shows what fraction of truly positive samples were correctly labeled as such; the precision, which shows what part of the positive predictions were correct; and the F1-score, which is the harmonic mean of the recall and precision \citep{sokolova2009systematic}. For each classifier, different configurations of hyperparameters (such as the kernel and the penalty-parameter) were systemically tested - each possible combination was tried. By saving the performance measures for every such configuration, for every classifier, each individual classifiers could be optimized. By then comparing the performance of every optimally configured classifier, the best-performing classifier could be chosen. 

Additionally, the added value of using a machine learning ensemble was measured. By having each classifier make a prediction for all videos in the dataset, a new dataset was created, wherein the features were the predictions made by the different classifiers. By using all possible combinations of classifiers, and then having different neural networks use those features as input, a machine learning ensemble was created. This ensemble was then optimized in a similar way to the classifiers individually.

\end{document}
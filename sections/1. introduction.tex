\documentclass[../main.tex]{subfiles}
\graphicspath{{\subfix{images/}}}

\begin{document}

\section{Introduction}
\subsection{Doelstelling en relevantie}
Het doel van dit onderzoek is om te kijken hoe snel (oftewel: na hoeveel gekeken video’s) het YouTube-aanbevelingsalgoritme een voorkeur krijgt voor complotvideo's. YouTube is met 34.6 miljard maandelijkse gebruikers de één na meestbezochte website op het internet, waardoor de impact van de website op de maatschappij niet onderschat kan worden \citep{neufeld_2021}. Er wordt content van alle categorieën geproduceerd en geconsumeerd. Echter, complotcontent begint een hoofdrol te spelen op YouTube. Alt-right (ook wel 'far-right') en complotkanalen krijgen een steeds grotere aanhang, wat negatieve gevolgen kan hebben voor de samenleving. Zo zijn er, mede dankzij complotcontent op YouTube, steeds meer mensen die beginnen te twijfelen aan de wetenschap. Wanneer zulke twijfels zich voordoen bij belangrijke onderwerpen, zoals het wel of niet innemen van het vaccin tegen het coronavirus, kan dit gevaarlijke gevolgen hebben voor de maatschappij. Zo is meer dan de helft van de Amerikaanse populatie twijfelachtig over - of zelf definitief tegen - het nemen van het coronavaccin \citep{rosenbaum2021escaping}. Dit onderzoek is gebaseerd op een aflevering van VPRO's \textit{Zondag met Lubach}, waarin eenzelfde idee op kleinere schaal werd uitgevoerd \citep{lubach_2020}. De uiterst interessante resultaten van dat experiment hebben mij gemotiveerd om het in grotere mate te onderzoeken. 

Er is al onderzoek gedaan naar filterbubbels en complot-content op YouTube, maar deze onderzoeken houden zich niet bezig met hoe snel een dergelijke bubbel ontstaat. Daarnaast kijken deze onderzoeken specifiek naar aanbevelingen op content, waarbij de aanbevelingen gebaseerd worden op de gelijkenis van de video’s en niet op het kijkgedrag van gebruikers. 

\subsection{Vraagstelling}
Het onderzoek zal zich richten op de volgende hoofdvraag:
\textit{Wat is de invloed van verschillende kijkstrategieën op het aantal complotvideos dat gekeken moet worden voordat de YouTube-aanbevelingen van een gebruiker de voorkeur krijgen voor conspiracy content?}

Hierbij zal de situatie als 'voorkeur' worden geteld zodra het percentage complotvideo's binnen de gebruikers aanbevelingen significant hoger is dan bij de baseline. Om deze hoofdvraag te beantwoorden, zullen er eerst drie deelvragen beantwoord moeten worden. Deze luiden als volgt:

\begin{itemize}
    \item Bij welke kijkstrategie komt een gebruiker het snelst in een filterbubbel van complotvideo's terecht op YouTube?
    \item Hoe lang duurt het voor een YouTube-gebruiker om uit een filterbubbel te komen, wanneer deze zich erin bevindt?
    \item Welk type classifier werkt het beste om complotvideo’s op YouTube te labelen?
\end{itemize}

\end{document}
\documentclass[../main.tex]{subfiles}
\graphicspath{{\subfix{images/}}}

\begin{document}

\section{Discussion}

\subsection{The filter bubbles}
% The results indicate that the YouTube algorithm is indeed susceptible to filter bubbles. Even with content that could be as harmful as conspiracy videos, the algorithm recommends more of what the user already watched. We live in a society. Link to previous literature!

\subsection{Cold-start problem}
% For each strategy, including the baseline, there was a clear example of a cold-start problem for the 
% algorithm \citep{lam2008addressing}. This became apparent through the average popularity of the recommended 
% content at the start of the experiment.

\subsection{Watch time optimization}
% The average duration of videos is higher for the conspiracy strategies, but it seems somewhat random. --> this probably is because longer videos do not necessarily lead to more watch time: a 10 minute video being watched in its entirety is better than a 45 minute video being skipped after 4 minutes. 


\end{document}
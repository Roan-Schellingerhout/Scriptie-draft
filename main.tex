\documentclass{uva-inf-article}

\usepackage[utf8]{inputenc}
\usepackage[english]{babel}
\usepackage{graphicx}
\graphicspath{{images/}}

\usepackage{natbib}
\usepackage{apalike}
\usepackage{booktabs}
\usepackage{lineno}
\usepackage{dblfloatfix}
\usepackage{subfiles}
\usepackage{adjustbox}
\usepackage{siunitx}
\usepackage{subcaption}
\usepackage[toc,page]{appendix}
\usepackage[linesnumbered,ruled,vlined]{algorithm2e}

\makeatletter
\patchcmd{\@startsection}{\@ifstar}{\nolinenumbers\@ifstar}{}{}
\patchcmd{\@xsect}{\ignorespaces}{\linenumbers\ignorespaces}{}{}
\makeatother

\bibliographystyle{apalike}
\setlength{\parindent}{10ex}

\assignment{Bachelor Thesis}
\title{YouTube's bias towards conspiracy content: an analysis of the YouTube algorithm}
\authors{Roan Schellingerhout}
\docent{Dr. Maarten Marx}
\date{\today}

\begin{document}

\maketitle

\begin{abstract}
    YouTube is the second largest website on the internet. It brings in over 34.6 billion page views each month,
    most of which consist of videos being watched. 70\% of those videos are found through YouTube's 
    recommendation algorithm. Unfortunately, this recommender system, like most, is sensitive to filter bubbles. In recent years, conspiracy content has been gaining popularity on the website, partially because of the way in which the algorithm handles it. To gain a better understanding of how the algorithm deals with conspiracies, an experiment was setup wherein brand new YouTube accounts were made to watch conspiracy content adhering to different watch strategies. After watching 15 conspiracy videos, all accounts ended up with significantly more conspiracy recommendations than the baseline. Accounts that watched conspiracy videos that were recommended to them by YouTube ended up in filter bubbles especially quickly. While accounts ended up in a filter bubble after watching between five and seven videos, getting out of a filter bubble took considerably longer. This suggests that YouTube's algorithm is indeed susceptible to the creation of filter bubbles of conspiracy content. 
\end{abstract}

{\bf Keywords:} Recommender systems, Conspiracy theories, Filter bubbles, YouTube, Machine learning

\linenumbers

\subfile{sections/1. introduction}

\subfile{sections/2. theoretical framework}

\subfile{sections/3. methodology}

\subfile{sections/4. results}

\subfile{sections/5. discussion}

\subfile{sections/6. conclusion}

\subfile{sections/7. future research}

\newpage

\subfile{sections/planning}

\newpage
\nolinenumbers

\bibliography{sources} 

\newpage

\subfile{sections/8. appendix}

\end{document}
